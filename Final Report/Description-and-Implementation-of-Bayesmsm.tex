% !TeX root = RJwrapper.tex
\title{Bayesmsm: An R package for longitunal causal analysis using Bayesian Marginal Structural Models}


\author{by Xiao Yan and Kuan Liu}

\maketitle

\abstract{%
Observational studies offer a viable, efficient, and low-cost design to readily gather evidence on exposure effects. Although more practical, exposure mechanism is nonrandomized and causal inference methods are required to draw causal conclusions. Popular approaches used in health research are predominantly frequentist methods. Bayesian approaches have unique estimation features that are useful in many settings, however, there is a general lack of open-access software packages to carry out these analyses. Our project seeks to address this gap by developing a user-friendly R package named ``bayesmsm'' for the implementation of the Bayesian Marginal Structural Models for longitudinal data with continuous or binary outcome.
}

\section{Introduction}\label{introduction}

Interactive data graphics provides plots that allow users to interact
them. One of the most basic types of interaction is through tooltips,
where users are provided additional information about elements in the
plot by moving the cursor over the plot.

This paper will first review some R packages on interactive graphics and
their tooltip implementations. A new package \CRANpkg{ToOoOlTiPs} that
provides customized tooltips for plot, is introduced. Some example plots
will then be given to showcase how these tooltips help users to better
read the graphics.

\section{Longitudinal causal framework and Bayesian Marginal Structural Models (BMSMs)}\label{longitudinal-causal-framework-and-bayesian-marginal-structural-models-bmsms}

\subsection{Notation}\label{notation}

In this section, we introduce the notation for the Bayesian Marginal Structural Models (BMSMs). Let \(n\) be the total number of subjects enrolled in the study, indexed by \(i = 1, \ldots, n\). Each subject is observed over \(J\) visits, indexed by \(j = 1, \ldots, J\). For each subject \(i\) at visit \(j\), let \(Y_{i}\) denote the final outcome, \(X_{ij}\) the covariates, and \(Z_{ij}\) the treatment. The full histories of these variables up to visit \(j\) for subject \(i\) are represented as \(Y_{i}\), \(\bar{X}_{ij} = \{X_{i1}, \ldots, X_{ij}\}\), and \(\bar{Z}_{ij} = \{Z_{i1}, \ldots, Z_{ij}\}\), respectively. At each visit \(j\), \(X_{ij}\) is measured first, and \(Z_{ij}\) is decided and recorded afterward (Liu 2021).

In the case of right-censoring, we introduce the censoring indicator \(C_{ij}\), where \(C_{ij} = 1\) if the subject \(i\) is censored at visit \(j\) and \(C_{ij} = 0\) if not censored. The full history of the censoring indicators up to visit \(j\) is denoted as \(\bar{C}_{ij} = \{C_{i1}, \ldots, C_{ij}\}\). The observed data for each subject \(i\) at visit \(j\) is thus represented by \(\bar{V}_{ij} = Y_{i}, \bar{X}_{ij}, \bar{Z}_{ij-1}, \bar{C}_{ij}\}\).

\subsection{Causal structure}\label{causal-structure}

Under the observational setting, the treatment decision at each visit \(j\) is made based on covariates \(X_{ij}\) and the past treatment assignment history \(\bar{Z}_{ij-1}\). We assume that the final outcome \(Y_{i}\) is influenced by the entire history of covariates and treatment assignments.

The hypothesized causal structure for two consecutive visits and a follow-up response is represented in a directed acyclic graph (DAG) in the next section. This structure assumes that there is no competition for treatment resources between subjects at each visit, meaning that the treatment assignment for one subject does not affect the outcome of another subject. Furthermore, we assume that there are no unmeasured confounders with respect to the outcome and the treatment at any given visit. This assumption is often referred to as the sequential ignorability of treatment assignment, which implies that given the observed history up to visit \(j\), the treatment assignment \(Z_{ij}\) is independent of the potential outcomes (Liu et al. 2020).

\subsection{Directed Acyclic Graph (DAG)}\label{directed-acyclic-graph-dag}

\label{fig:figs}Example DAG for 2 visits

Figure 1 shows the causal structure for two consecutive visits with a final outcome , which is
the Directed Acyclic Graph (DAG) between treatment assignment,
covariates and the final outcome. A causal graph is a directed acyclic
graph in which the vertices (nodes) of the graph represent variables and
the directed edges (arrows) represent direct causal effects
(Robins et al. 2000). Here, \(X_{ij-1}\) is the baseline
covariate. The covariates \(X_{ij-1}\), \(X_{ij}\) and time-varying covariate
\(Z_{ij-1}\) potentially influence treatment assignments \(Z_{ij}\), and
ultimately, the outcome \(Y_{i}\). Each treatment decision \(Z_j\) at visit \(j\)
is determined at the end of that visit, based on all previous
information. This DAG also assumes that there is no unmeasured
confounding.

\subsection{Data-generating mechanism causal framework}\label{data-generating-mechanism-causal-framework}

We construct a data-generating mechanism for a
non-repeatedly measured outcome \(Y_{i}\). Under the data-generating
mechanism, causal inference with observational data can simply be viewed
as a prediction problem (Liu et al. 2020). This framework allows us to conceptualize the problem as drawing inference from an ideal population, where treatment assignment is unconfounded (similar to a randomized setting), and comparing it to the data observed in a confounded observational setting (Dawid, Didelez, et al. 2010; Røysland et al. 2011; Arjas 2012; Saarela et al. 2015; Hernán and Robins 2016). The experimental data generating mechanism is
indexed by \(\mathcal{E}\), which generates samples from the ideal
population where treatment assignment at each visit is independent of
the covariates given past treatment assignments, i.e.,
\(Z_{ij} \perp X_{ij} \mid Z_{ij-1}\); the observational data generating
mechanism is indexed by \(\mathcal{O}\), which generates samples from the
observed population where independence does not hold and treatment
assignment \(Z_{ij}\) at visit \(j\) depends on \(\{X_{ij}, Z_{ij-1}\}\) (Liu et al. 2020).

Under such data-generating mechanism, we have two important assumptions.
The first assumption has already been stated above, which is
\(Z_{ij} \perp X_{ij} \mid Z_{ij-1}\). The second positivity assumption
states that at each visit, any treatment sequence that is compatible
with the complete treatment history has a non-zero probability of
occuring, i.e.~\(0 < P(Z_{ij} | X_{ij}, Z_{ij-1}) < 1\) for
\(j = 1, \ldots, J\).

Under \(\mathcal{E}\), we can specify the marginal outcome model as
(Liu 2021):

\[
   g_y( E_{\mathcal{E}}[{y}_{i} \mid \bar{z}_{iJ}]) = {\Theta} \
   \sum_{j=1}^J z_{ij}, \quad j=1,\ldots,J.
\]

We draw inference from an ideal population (\(\mathcal{E}\)) but with
observed data subject to confounding (\(\mathcal{O}\)). Here, \({\Theta}\)
is known as the marginal treatment effect. For example, if we have a
binary time-varying treatment, \({\Theta}\) is interpreted as the expected change in the final outcome \(Y_i\) relative to a one-unit increase in the cumulative treatment exposure prior to visit \(j\).

\subsection{Bayesian Framework}\label{bayesian-framework}

Let \(Y\) represent the outcome variable, \(A\) the matrix of treatment
variables, and \(W\) the vector of patient weights. The number of
observations is \(n\), and \(\theta\) is the vector of causal parameters on
the mean, with \(\sigma\) representing the standard deviation. The
weighted log-likelihood of the normal distribution is given by
\cite{shaliziLecture}:

\[
\text{weighted log}\mathcal{L}_{\text{normal}}(\theta, \sigma^2) = -\frac{n}{2} \log(\sigma^2) - \frac{1}{2\sigma^2} \sum_{i=1}^{n} W_i (Y_i - A_i \theta)^2
\]

where \(A_i\) is the \(i\)-th row of matrix \(A\) and \(W_i\) denotes the
treatment weight for the \(i\)-th patient.

Using Bayesian decision theory and importance sampling technique, we
maximize an expected utility function (a function involving only
\(\theta\)), \(\textbf{u}_{\mathcal{E}}(\Theta, \bar{v}_{i}^*)\), via
posterior predictive inference \cite{liuBayesianCausal},

\[
\hat{\Theta} 
 = \text{argmax}_{\theta} \int_{\bar{v}_{i}^*}  u_{\mathcal{E}}(\Theta, \bar{v}_{i}^*)P_{\mathcal{E}}(\bar{v}_{i}^* \mid \textbf{V}_n) \ d\bar{v}_{i}^* 
= \text{argmax}_{\theta}\int_{\bar{v}_{i}^*}  u_{\mathcal{E}}(\Theta, \bar{v}_{i}^*) \frac{P_{\mathcal{E}}(\bar{v}_{i}^* \mid \textbf{V}_n) }{P_{\mathcal{O}}(\bar{v}_{i}^* \mid \textbf{V}_n)}P_{\mathcal{O}}(\bar{v}_{i}^* \mid \textbf{V}_n) \ d\bar{v}_{i}^* 
\]

where
\(u(\Theta, \bar{v}_{i}^*)= \log P_{\mathcal{E}}( Y_{i}^* \mid \bar{z}_{iJ}^*; \Theta)\)
is the utility function; and
\(w_{i}^* = \frac{P_{\mathcal{E}}(\bar{v}_{i}^* \mid \textbf{v}_n)}{P_{\mathcal{O}}(\bar{v}_{i}^* \mid \textbf{v}_n)}\)
can be expanded to treatment assignment weight \cite{liuBayesianCausal}:

\[
w_{i j}^{*}=\frac{E_{\alpha}\left[\prod_{l=1}^{j} P_{\mathcal{E}}\left(Z_{i l-1}^{*} \mid \bar{Z}_{i l-2}^{*}, \alpha_{l-1}\right) \mid \bar{z}_{1}, \ldots, \bar{z}_{n}\right]}{E_{\beta}\left[\prod_{l=1}^{j} P_{\mathcal{O}}\left(Z_{i l-1}^{*} \mid \bar{Z}_{i l-2}^{*}, \bar{X}_{i l-1}^{*}, \beta_{l-1}\right) \mid \mathbf{v}_{n}\right]}, \text { for } j=1, \ldots, k \text {. }
\]

Since the Bayesian decision argument offers a point estimate for the
marginal treatment effect where the uncertainty is not yet addressed, we
need to use the weighted likelihood bootstrap (WLB) method
\cite{liuBayesianCausal}. This WLB method allows us to obtain a
distribution of \(\Theta\) by drawing \(\pi\) from a uniform Dirichlet
distribution of and maximizing the \(\pi\) weighted expected utility sum
\(\sum_{i=1}^{n} \sum_{j=1}^{k+1} \pi_{i}^{(b)} w_{i j} \log P_{\mathcal{E}}\left(y_{i j} \mid \bar{z}_{i j-1} ; \Theta\right)\)
with respect to \(\Theta\) \cite{liuBayesianCausal}.

In applying the above methods to actual coding, we first code the
weighted log likelihood of normal distribution in R. In order for the
automation of sampling from the uniform Dirichlet distribution and then
maximizing the likelihood, we initialize three mean parameters to 0.1
and the variance parameter to 4. Then, in each bootstrap, we draw 1,000
samples of alpha, the size of the dataset, from Dirichlet distribution
with parameter alpha equal to 1. The weighted mean is set to 1 for each
observation. Finally, we maximize the weighted log likelihood function.

The Average Causal Effect (ACE), or the Average Treatment Effect (ATE),
is defined as \cite{dingFirstCourse}:

\[
  \tau = n^{-1} \sum_{i=1}^{n} \{Y_i(1) - Y_i(0)\} \\
  = n^{-1} \sum_{i=1}^{n} Y_i(1) - n^{-1} \sum_{i=1}^{n} Y_i(0).
\]

In this study, the causal parameter of interest is the ATE between
always treated vs never treated \cite{liu2023section3}. After getting
the Average Treatment Effect (ATE) from Bayesian bootstrap, we will also
compare our results to the frequentist approach. The frequentist
Marginal Structural Models (MSMs) calculates visit specific propensity
scores, and fits the weighted linear regression \cite{liu2023section3}.

\section{Description of bayesmsm}\label{description-of-bayesmsm}

The \texttt{bayesmsm} package is developed to implement Bayesian marginal
structural models (BMSMs) for longitudinal data analysis. It contains
three core functions: \texttt{bayesweight}, \texttt{bayesweight\_cen}, and \texttt{bayesmsm}.
\texttt{bayesweight} estimates treatment weights using posterior samples of
\(\alpha\) and \(\beta\) via fitting a series of logistic regressions in a
Bayesian framework, whereas \texttt{bayesweight\_cen} extends the function
\texttt{bayesweight} to handle right-censored data. \texttt{bayesmsm} then uses the
estimated treatment weights to perform Bayesian non-parametric bootstrap
so as to estimate causal effect. In this section, we describe these
functions and their usage in detail.

\subsection{\texorpdfstring{Bayesian treatment effect weight estimation using \texttt{bayesweight}}{Bayesian treatment effect weight estimation using bayesweight}}\label{bayesian-treatment-effect-weight-estimation-using-bayesweight}

\begin{itemize}
\tightlist
\item
  The following code calls the function \texttt{bayesweight} to run JAGS and
  calculate the weights.

  \begin{itemize}
  \tightlist
  \item
    Non-parallel computing requires that \texttt{n.chains\ =\ 1}. Parallel
    MCMC requires at least 2 chains because computing is running on
    1 core per chain, and we recommend using at most 2 chains less
    than the number of available cores on your computer.
  \item
    Running this function automatically saves a JAGS model file in
    the working directory, which the user can check to review the
    model specifications.
  \end{itemize}
\item
  Parameters Description:

  \begin{itemize}
  \tightlist
  \item
    \texttt{trtmodel.list}: A list of formulas corresponding to each time
    point with the time-specific treatment variable on the left hand
    side and pre-treatment covariates to be balanced on the right
    hand side. Interactions and functions of covariates are allowed.
  \item
    \texttt{data}: The dataset containing all the variables specified in
    trtmodel.list.
  \item
    \texttt{n.iter}: Total number of iterations for each chain (including
    burn-in).
  \item
    \texttt{n.burnin}: Number of iterations to discard at the beginning of
    the simulation (burn-in).
  \item
    \texttt{n.thin}: Thinning rate for the MCMC sampler.
  \item
    \texttt{n.chains}: Number of MCMC chains to run. For non-parallel
    execution, this should be set to 1. For parallel execution, it
    requires at least 2 chains.
  \item
    \texttt{seed}: Seed to ensure reproducibility.
  \item
    \texttt{parallel}: Logical flag indicating whether to run the MCMC
    chains in parallel. Default is TRUE.
  \end{itemize}
\end{itemize}

\texttt{weights\ \textless{}-\ bayesweight(trtmodel.list\ =\ list(a\_1\ \textasciitilde{}\ w1\ +\ w2\ +\ L1\_1\ +\ L2\_1,}

\texttt{a\_2\ \textasciitilde{}\ w1\ +\ w2\ +\ L1\_1\ +\ L2\_1\ +\ L1\_2\ +\ L2\_2\ +\ a\_1),}

\texttt{data\ =\ testdata,}

\texttt{n.iter\ =\ 250,}

\texttt{n.burnin\ =\ 150,}

\texttt{n.thin\ =\ 5,}

\texttt{n.chains\ =\ 2,}

\texttt{seed\ =\ 890123,}

\texttt{parallel\ =\ TRUE)}

\begin{itemize}
\tightlist
\item
  It returns a list containing:

  \begin{itemize}
  \tightlist
  \item
    \texttt{weights}: The calculated weights for subject-specific treatment
    effects.
  \end{itemize}
\end{itemize}

\subsection{\texorpdfstring{Bayesian non-parametric bootstrap to maximize the utility function with respect to the causal effect using \texttt{bayesmsm}}{Bayesian non-parametric bootstrap to maximize the utility function with respect to the causal effect using bayesmsm}}\label{bayesian-non-parametric-bootstrap-to-maximize-the-utility-function-with-respect-to-the-causal-effect-using-bayesmsm}

The function \texttt{bayesmsm} estimates causal effect of time-varying
treatments. It uses subject-specific treatment assignmennt weights
\emph{weights} calculated using \texttt{bayesweight} or \texttt{bayesweight\_cen}, and
performs Bayesian non-parametric bootstrap to estimate the causal
parameters.

\begin{itemize}
\tightlist
\item
  Parameters Description:

  \begin{itemize}
  \tightlist
  \item
    \texttt{ymodel}: A formula representing the outcome model, which can
    include interactions and functions of covariates.
  \item
    \texttt{nvisit}: Specifies the number of visits or time points
    considered in the model.
  \item
    \texttt{reference}: The baseline or reference intervention across all
    visits, typically represented by a vector of zeros indicating no
    treatment (default is a vector of all zeros).
  \item
    \texttt{comparator}: The comparison intervention across all visits,
    typically represented by a vector of ones indicating full
    treatment (default is a vector of all ones).
  \item
    \texttt{family}: Specifies the outcome distribution family; use
    ``gaussian'' for continuous outcomes or ``binomial'' for binary
    outcomes (default is ``gaussian'').
  \item
    \texttt{data}: The dataset containing all variables required for the
    model.
  \item
    \texttt{wmean}: A vector of treatment assignment weights. Default is a
    vector of ones, implying equal weighting.
  \item
    \texttt{nboot}: The number of bootstrap iterations to perform for
    estimating the uncertainty around the causal estimates.
  \item
    \texttt{optim\_method}: The optimization method used to find the best
    parameters in the model (default is `BFGS').
  \item
    \texttt{seed}: A seed value to ensure reproducibility of results.
  \item
    \texttt{parallel}: A logical flag indicating whether to perform
    computations in parallel (default is TRUE).
  \item
    \texttt{ncore}: The number of cores to use for parallel computation
    (default is 4).
  \end{itemize}
\end{itemize}

\texttt{model\ \textless{}-\ bayesmsm(ymodel\ =\ y\ \textasciitilde{}\ a\_1+a\_2,}
\texttt{nvisit\ =\ 2,}
\texttt{reference\ =\ c(rep(0,2)),}
\texttt{comparator\ =\ c(rep(1,2)),}
\texttt{family\ =\ "gaussian",}
\texttt{data\ =\ testdata,}
\texttt{wmean\ =\ weights,}
\texttt{nboot\ =\ 1000,}
\texttt{optim\_method\ =\ "BFGS",}
\texttt{parallel\ =\ TRUE,}
\texttt{seed\ =\ 890123,}
\texttt{ncore\ =\ 2)}

\begin{itemize}
\tightlist
\item
  It returns a model object which contains:

  \begin{itemize}
  \tightlist
  \item
    \texttt{mean}, \texttt{sd}, \texttt{quantile}: the mean, standard deviation and 95\%
    credible interval of the estimated causal effect (ATE). From the
    above results, the mean of ATE is approximately -3.161, which
    indicates that the expected outcome for always treated patients
    is, on average, 3.161 units less than that for never treated
    patients.
  \item
    \texttt{bootdata}: a data frame containing the bootstrap samples for
    the reference effect, comparator effect, and average treatment
    effect (ATE).
  \item
    \texttt{reference}, \texttt{comparator}: the reference level and comparator
    level the user chooses to compare. Here the reference level is
    never treated (0,0), and the comparator level is always treated
    (1,1).
  \end{itemize}
\end{itemize}

\section{Numerical Examples and Implementation Using Simulated Dataset}\label{numerical-examples-and-implementation-using-simulated-dataset}

In our simulation study, we will be using a longitudinal dataset
designed and simulated to mimic the complex real-world clinical data
\cite{liu2023section3}. In this dataset, there are 1,000 patients in
total with 3 visits, 2 of which patients were assigned a treatment. The
end-of-study outcome \(Y\) is continuous and what we are interested in.
\(w_1\) and \(w_2\) are two baseline covariates mimicking sex and age, where
\(w_1\) is a binary variable with values 0 (female) and 1 (male), and
\(w_2\) is a continuous variable with mean 12.048. \(L_1\) and \(L_2\) emulate
time-dependent covariates, one binary and one continuous, reflecting
variables that might change with time during the study period. The
binary treatment variable, represented as \(a_1\) and \(a_2\) for visit 1
and 2 in the dataset, corresponds to the treatments \(Z_1\) and \(Z_2\) as
defined in our earlier discussion on Directed Acyclic Graphs (DAGs).
Finally, there is no missing data in the dataset. \cite{liu2023section3}
Table 1 shows an overview of this simulated dataset.

\begin{itemize}
\tightlist
\item
  The simulated DAG
\end{itemize}

\begin{itemize}
\tightlist
\item
  Frequency Counts by Treatment Combinations
\end{itemize}

\begin{verbatim}
#>    
#>       0   1
#>   0 520 201
#>   1 111 168
\end{verbatim}

\begin{itemize}
\tightlist
\item
  Suppose the causal parameter of interest is the average treatment
  effect between always treated and never treated,
\end{itemize}

\[
ATE = E(Y \mid Z_1 = 1, Z_2 = 1) - E(Y \mid Z_1 = 0, Z_2 = 0)
\]

\begin{itemize}
\tightlist
\item
  Usage of functions on the simulated dataset:
\end{itemize}

\begin{verbatim}
#>  num [1:1000] 1.257 1.114 1.022 0.85 0.806 ...
\end{verbatim}

\begin{verbatim}
#> List of 6
#>  $ RD_mean    : num -3.16
#>  $ RD_sd      : num 0.0947
#>  $ RD_quantile: Named num [1:2] -3.34 -2.99
#>   ..- attr(*, "names")= chr [1:2] "2.5%" "97.5%"
#>  $ bootdata   :'data.frame': 1000 obs. of  3 variables:
#>   ..$ effect_reference : num [1:1000] 2.36 2.29 2.26 2.34 2.31 ...
#>   ..$ effect_comparator: num [1:1000] -0.947 -0.824 -0.757 -0.797 -0.765 ...
#>   ..$ RD               : num [1:1000] -3.31 -3.11 -3.02 -3.13 -3.08 ...
#>  $ reference  : num [1:2] 0 0
#>  $ comparator : num [1:2] 1 1
\end{verbatim}

\section{Discussion}\label{discussion}

\section*{References}\label{references}
\addcontentsline{toc}{section}{References}

\phantomsection\label{refs}
\begin{CSLReferences}{1}{0}
\bibitem[\citeproctext]{ref-arjas2012causal}
Arjas, E. 2012. \emph{Causal Inference from Observational Data: A Bayesian Predictive Approach}. Wiley Online Library. \url{https://onlinelibrary.wiley.com/doi/abs/10.1002/9781119945710.ch7}.

\bibitem[\citeproctext]{ref-dawid2010identifying}
Dawid, AP, V Didelez, et al. 2010. {``Identifying the Consequences of Dynamic Treatment Strategies: A Decision-Theoretic Overview.''} \emph{Statistical Surveys} 4: 184--231.

\bibitem[\citeproctext]{ref-hernanRobins2016}
Hernán, MA, and JM Robins. 2016. {``Using Big Data to Emulate a Target Trial When a Randomized Trial Is Not Available.''} \emph{American Journal of Epidemiology} 183 (8): 758--64.

\bibitem[\citeproctext]{ref-liuRepeatedlyMeasured}
Liu, K. et al. 2020. {``Estimation of Causal Effects with Repeatedly Measured Outcomes in a Bayesian Framework.''} \emph{Statistical Methods in Medical Research}.

\bibitem[\citeproctext]{ref-liuBayesianCausal}
Liu, K. 2021. {``Bayesian Causal Inference with Longitudinal Data.''}

\bibitem[\citeproctext]{ref-robinsMarginalStructural}
Robins, J. et al. 2000. {``Marginal Structural Models and Causal Inference in Epidemiology.''}

\bibitem[\citeproctext]{ref-roysland2011martingale}
Røysland, K et al. 2011. {``A Martingale Approach to Continuous-Time Marginal Structural Models.''} \emph{Bernoulli} 17 (3): 895--915.

\bibitem[\citeproctext]{ref-saarelaBayesianEstimation}
Saarela, O. et al. 2015. {``On Bayesian Estimation of Marginal Structural Models.''}

\end{CSLReferences}


\address{%
Xiao Yan\\
Dalla Lana School of Public Health, University of Toronto\\%
Department of Biostatistics\\ Toronto, Canada\\
%
\url{https://github.com/XiaoYan-Clarence}\\%
\textit{ORCiD: \href{https://orcid.org/0000-1721-1511-1101 (?)}{0000-1721-1511-1101 (?)}}\\%
\href{mailto:Clarence.YXA@gmail.com}{\nolinkurl{Clarence.YXA@gmail.com}}%
}

\address{%
Kuan Liu\\
Dalla Lana School of Public Health, University of Toronto\\%
Department of Biostatistics, Toronto, Canada\\ Institute of Health Policy, Management and Evaluation, Toronto, Canada\\
%
\url{https://www.kuan-liu.com/}\\%
\textit{ORCiD: \href{https://orcid.org/0000-0002-5017-1276}{0000-0002-5017-1276}}\\%
\href{mailto:kuan.liu@utoronto.ca}{\nolinkurl{kuan.liu@utoronto.ca}}%
}
